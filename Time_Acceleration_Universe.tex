
\documentclass[12pt]{article}
\usepackage[margin=1in]{geometry}
\usepackage{amsmath}
\usepackage{graphicx}
\usepackage{hyperref}
\usepackage{longtable}

\title{On the Acceleration of Time with Metric Expansion of the Universe}
\author{Viktor Iassinskii, PhD\\Independent Researcher\\\texttt{iassinskii@gmail.com}}
\date{}

\begin{document}

\maketitle

\begin{abstract}
This paper presents a hypothesis rooted in the framework of the Theory of Relativity: that the flow of time is not uniform across cosmic history, but instead accelerates in proportion to the expansion of the universe. Building on the concept that space and time are intertwined within the fabric of spacetime, and recognizing that the speed of light remains constant while the metric of space evolves, we propose that the subjective "tick rate" of time is proportional to the cosmological scale factor $a(t)$. Under this model, one second in the early universe may equate to billions of seconds today. We explore this idea using simplified relationships derived from relativistic cosmology and compare how "one second" scales across epochs including the Planck era, recombination, and the present. Though not derived from quantum gravity, this model presents a heuristic framework for considering the evolution of temporal flow in an expanding universe.
\end{abstract}

\section{Introduction}
The prevailing cosmological model describes a universe born from a singular event---the Big Bang---which has been expanding ever since. While the metric expansion of space is well understood through General Relativity and supported by observational data such as redshift, the nature of time remains more elusive. Time is usually treated as a passive backdrop or a coordinate within the four-dimensional spacetime manifold, evolving uniformly and independently of the expansion of space.

This paper proposes a simple but provocative hypothesis: that the flow of time is not uniform, but instead accelerates as the universe expands, in direct proportion to the cosmological scale factor $a(t)$. Under this model, the experience or "tick rate" of one second is not fixed across cosmic history. What we call a second today may have encompassed vastly more subjective or causal depth in earlier epochs, particularly in the Planck era or near recombination.

Just as the curvature of the Earth is imperceptible at local scales yet defines the planet's global geometry, so too might the evolution of time be imperceptible over human lifespans, while undergoing substantial transformation over cosmic time. This paper explores this analogy and its implications for both fundamental physics and cosmological observation.

\section{Background}
In modern physics, time and space are not treated as separate entities, but as interwoven components of a four-dimensional continuum known as spacetime. Introduced by Einstein’s Theory of Relativity, this framework radically altered our understanding of how time behaves under the influence of gravity and motion. In particular, General Relativity reveals that time can dilate in strong gravitational fields or when objects approach the speed of light.

The reason for this is that gravity is not a traditional force in Einstein's framework, but rather the result of curved spacetime. Massive objects, like planets or black holes, curve the space around them. The stronger the gravity, the greater the curvature. Because space and time are deeply linked, this curvature slows the passage of time. A clock closer to a massive object (deeper in a gravitational well) will tick more slowly than a clock farther away.

This phenomenon, known as gravitational time dilation, has been confirmed experimentally through atomic clock measurements on satellites and Earth, and it is routinely corrected for in technologies like GPS.

Cosmology extends these relativistic principles to the largest scales, modeling the universe as a dynamic entity governed by the Friedmann-Lemaître-Robertson-Walker (FLRW) metric. This model describes an expanding universe in which spatial distances evolve according to a dimensionless function known as the scale factor $a(t)$, which grows with time since the Big Bang.

This creates a natural consistency: in regions of high density or curvature (e.g., near black holes or in the early universe), time slows down. As the universe expands and overall curvature weakens, time accelerates. The model presented in this paper extends this relationship to a cosmic scale, proposing that the rate of time flow itself evolves with the geometry of the universe.

While the scale factor captures the expansion of space, time is typically assumed to progress uniformly, with "seconds" considered constant and invariant in duration. However, this assumption may warrant re-examination. If space expands, and space and time are inseparably linked, might time itself evolve---not just in terms of coordinate transformations, but in its rate of flow?

\section{Hypothesis and Temporal Flow Model}
We propose the following hypothesis:

\textit{The flow rate of time accelerates in direct proportion to the expansion of space, such that the duration of one second varies across cosmic epochs according to the cosmological scale factor $a(t)$.}

We define a Time Flow Factor $T(t)$, which governs how rapidly time "ticks" at a given cosmic time $t$. We define:
\[
T(t) \propto a(t)
\]

This implies that the rate at which moments are experienced—or the subjective tempo of time—increases as the universe expands. Consequently, the experienced duration of one second in earlier epochs is inversely related to the scale factor:
\[
\text{Perceived Duration of 1 Second at Time } t = \frac{1}{T(t)} \propto \frac{1}{a(t)}
\]

\begin{itemize}
    \item In the early universe, when $a(t) \ll 1$, one second spanned more "temporal substance"—it was stretched and sparse.
    \item Today, with $a(t) \approx 1$, time flows at its current, familiar rate.
    \item In the far future, with $a(t) > 1$, time will become denser—more events or "ticks" can occur per unit time.
\end{itemize}

\section{Temporal Comparison Across Cosmic Epochs}

\begin{center}
\begin{tabular}{|p{3.2cm}|p{3.2cm}|p{2.5cm}|p{2.5cm}|p{3.2cm}|}
\hline
\textbf{Epoch} & \textbf{Time Since Big Bang} & \textbf{Scale Factor $a(t)$} & \textbf{Time Flow $T(t)$} & \textbf{1 Second Feels Like} \\
\hline
Planck Epoch & $\sim 10^{-43}$ s & $\sim 10^{-32}$ & $\sim 10^{-32}$ & $\sim 10^{32}$ seconds \\
Recombination (CMB) & $\sim 380,000$ years & $\sim 10^{-3}$ & $\sim 10^{-3}$ & 1,000 seconds \\
100 Years Ago & $\sim 13.8$ billion years & $\sim 0.9999999927$ & $\sim 0.9999999927$ & 1.0000000073 seconds \\
Now & Present & $1$ & $1$ & 1 second \\
\hline
\end{tabular}
\end{center}

\section{Discussion}
This model suggests that, as the universe expands, the rate of time flow accelerates, with the most dramatic differences occurring at early cosmic times. The Planck Epoch represents a period where time, as we understand it, would be almost inconceivable—a second then would stretch into millions of years in today’s terms.

The analogy with Earth's curvature illustrates the subtlety of time acceleration. Locally, the change in time flow is minuscule—much like a person walking across a field does not perceive the Earth as round. However, over vast scales, just as Earth's curvature becomes measurable, the evolving flow of time may reveal itself in cosmological measurements.

One particularly intriguing implication of this hypothesis relates to the observability of the universe’s farthest regions. Standard cosmology holds that certain parts of the universe are forever beyond our view because they are receding faster than the speed of light due to the expansion of space. However, if the rate at which we experience time accelerates, it is conceivable that our frame of temporal perception could "catch up," allowing us to observe regions previously thought unreachable. This reinterpretation of the cosmic horizon, if valid, could have profound implications for both cosmology and the limits of observation.

\section{Conclusion}
The acceleration of time as the universe expands introduces a new perspective on the nature of time. By modeling time's flow in direct proportion to the expansion of space, we provide an intuitive framework for understanding how temporal experiences evolve across the epochs of the universe. While this idea awaits validation from a unified theory of quantum gravity, it offers a compelling conceptual model that bridges relativity, cosmology, and philosophical inquiry.

\section*{References}
\begin{enumerate}
    \item Einstein, A. (1916). \textit{The Foundation of the General Theory of Relativity}. Annalen der Physik.
    \item Friedmann, A. (1922). \textit{On the Curvature of Space}. Zeitschrift für Physik.
    \item Liddle, A. (2015). \textit{An Introduction to Modern Cosmology}. Wiley.
    \item Planck Collaboration. (2020). \textit{Planck 2018 results. VI. Cosmological parameters}. Astronomy \& Astrophysics.
    \item Misner, C. W., Thorne, K. S., \& Wheeler, J. A. (1973). \textit{Gravitation}. W. H. Freeman.
    \item Barbour, J. (1999). \textit{The End of Time: The Next Revolution in Physics}. Oxford University Press.
\end{enumerate}

\end{document}
